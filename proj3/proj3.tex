\documentclass[11pt, ]{article}
\usepackage[czech]{babel}
\usepackage[utf8]{inputenc}
\usepackage{times}
\usepackage[czech, ruled, linesnumbered, noline, longend]{algorithm2e}
\usepackage{graphicx}
\graphicspath{./pictures/}
\usepackage{pdflscape}
\usepackage{multicol}
\usepackage{multirow}
\usepackage{geometry}
\geometry{
a4paper,
total={17cm,24cm},
left=2cm,
top=3cm,
}
\usepackage[unicode]{hyperref}
\hypersetup{
colorlinks=false
}

\begin{document}
\begin{titlepage}
\begin{center}
\Huge
\textsc{Vysoké učení technické v Brně \\
\huge
Fakulta informačních technologií}\\[0.4em]
\vspace{\stretch{0.382}}
\LARGE
Typografie a publikování - 3. projekt\\[0.3em]
\Huge
Tabulky a obrázky
\vspace{\stretch{0.618}}
\end{center}
\large29.3.2020 \hfill Vojtěch Bůbela

\end{titlepage}

\section{Úvodní strana}

Název práce umístěte do zlatého řezu a nezapoměňte uvést dnešní datum a vaše jméno a příjmení

\section{Tabulky}
Pro sázení tabulek můžeme použít buď prostředí \verb!tabbing! nebo prostředí \verb!tabular!.

\subsection{Prostředí \texttt{tabbing}}

Při použití \verb!tabbing! vypadá tabulka následovně:

\begin{tabbing}
\hspace{3cm}               \= \hspace{1.5cm}    \= \hspace{2cm}    \kill
\textbf{Ovoce}             \> \textbf{Cena}     \> \textbf{Množství}    \\ 
Jablka                     \> 25.90             \> 3 kg                 \\
Hrušky                     \> 27.40             \> 2.5 kg               \\
Vodní melouny              \> 35,--             \> 1 kus                \\
\end{tabbing}
Toto prostředí se dá také použít pro sázení algoritmů, ovšem vhodnější je použít 
prostředí \verb!algorithm! nebo \verb!algorithm2e! (viz sekce 3).


\subsection{Prostředí \texttt{tabular}}

Další možností, jak vytvořit tabulku, je použít prostředí \verb!tabular!. Tabulky pak 
budou vypadat takto\footnote{Kdyby byl problémm s \texttt{cline} , zkuste se podívat třeba sem  \url{http://www.abclinuxu.cz/tex/poradna/show/325037}.\vfill{}}:

\begin{table}[h]
\catcode`\-=12
    \centering
    \begin{tabular}{|c|c|c|}
    \hline
                    & \multicolumn{2}{c|}{\textbf{Cena}}    \\\cline{2-3}
    \textbf{Měna}   & \textbf{nákup}    & \textbf{prodej}   \\\hline
    EUR             & 25,475            & 27,045            \\
    GBP             & 28,835            & 30,705            \\
    USD             & 22,943            & 24,357            \\
    
    \hline
    \end{tabular}
    \caption{Tabulka kurzů k dnešnímu dni}
    \label{tab:1}
\end{table}

\begin{table}[h!]
\catcode`\-=12
    \centering
    \begin{tabular}{|c|c|}
        \hline
        $A$    & $\neg A$ \\
        \hline
        \textbf{P}    & N \\
        \textbf{O}    & O \\
        \textbf{X}    & X \\
        \textbf{N}    & P \\
        \hline
    \end{tabular}
    \begin{tabular}{|c|c|c|c|c|c|}
        \hline
        \multicolumn{2}{|c|}{\multirow{2}{*}{$A \wedge B$}} & \multicolumn{4}{c|}{$B$}\\ \cline{3-6}
        \multicolumn{2}{|c|}{}&\textbf{P}&\textbf{O}&\textbf{X}&\textbf{N}\\
        \hline
        \multirow{4}{*}{$$A$$}
        &\textbf{P} &    P   &   O   &   X   &   N\\
        \cline{2-6}
        &\textbf{O} &    O   &   O   &   N   &   N\\
        \cline{2-6}
        &\textbf{X} &    X   &   N   &   X   &   N\\
        \cline{2-6}
        &\textbf{N} &    N   &   N   &   N   &   N\\
        \hline
    \end{tabular}
    \begin{tabular}{|c|c|c|c|c|c|}
        \hline
        \multicolumn{2}{|c|}{\multirow{2}{*}{$A \vee B$}} & \multicolumn{4}{c|}{$B$}\\ \cline{3-6}
        \multicolumn{2}{|c|}{}&\textbf{P}&\textbf{O}&\textbf{X}&\textbf{N}\\
        \hline
        \multirow{4}{*}{$$A$$}
        &\textbf{P}  &   P   &   P   &   P   &   P\\
        \cline{2-6}
        &\textbf{O}  &   P   &   O   &   P   &   O\\
        \cline{2-6}
        &\textbf{X}  &   P   &   P   &   X   &   X\\
        \cline{2-6}
        &\textbf{N}  &   P   &   O   &   X   &   N\\
        \hline
    \end{tabular}
    \begin{tabular}{|c|c|c|c|c|c|}
        \hline
        \multicolumn{2}{|c|}{\multirow{2}{*}{$A \rightarrow B$}} & \multicolumn{4}{c|}{$B$}\\ \cline{3-6}
        \multicolumn{2}{|c|}{}&\textbf{P}&\textbf{O}&\textbf{X}&\textbf{N}\\
        \hline
        \multirow{4}{*}{$$A$$}
        &\textbf{P}  &   P   &   O   &   X   &   N\\
        \cline{2-6}
        &\textbf{O}  &   P   &   O   &   P   &   O\\
        \cline{2-6}
        &\textbf{X}  &   P   &   P   &   X   &   X\\
        \cline{2-6}
        &\textbf{N}  &   P   &   P   &   P   &   P\\
        \hline
    \end{tabular}
    \caption{Protože Kleeneho trojhodnotová logika už je „zastaralá“, uvádíme si zde příklad čtyřhodnotové logiky}
    \label{tab:2}
\end{table} 



\section{Algoritmy}

Pokud budeme chtít vysázet algoritmus, můžeme použít prostředí \verb!algorithm!\footnote{Pro nápovědu, jak zacházet s prostředím \texttt{algorithm}, můžeme zkusit tuhle stránku:\newline
\url{http://ftp.cstug.cz/pub/tex/CTAN/macros/latex/contrib/algorithms/algorithms.pdf}.}
nebo \verb!algorithm2e!\footnote{Pro algorithm2e zase tuhle:
\url{http://ftp.cstug.cz/pub/tex/CTAN/macros/latex/contrib/algorithm2e/algorithm2e.pdf}.}.
Příklad použití prostředí algorithm2e viz Algoritmus \ref{algorithm:1}.

\begin{algorithm}[H] \label{algorithm:1}
\DontPrintSemicolon
\LinesNumbered
\SetNlSty{}{}{:}
\KwIn{$(X_{t-1}, u_t, z_t)$}
\KwOut{$X_t$}
$\overline{X_t} = X_t = 0$\;
\For{k = 1 to M}{
    $x^{[k]}_t = sample\_motion\_model (u_t,x^{[k]}_{t-1})$\;
    $\omega^{[k]}_t = measurement\_model(z_t,x^{[k]}_t,m_{t-1})$\;
    $m^{[k]}_t = updated\_occupancy\_grid(z_t,x^{[k]}_t,m^{[k]}_{t-1})$\;
    $\overline{X_t} = \overline{X_t} + \langle x^{[m]}_x, \omega^{[m]}_t \rangle$\;
}
\For{k = 1 to M}{
    draw $i$ with probability $\approx \omega^{[i]}_t$\;
    add $\langle x^{[k]}_x, \omega^{[k]}_t \rangle$ to $\overline{X_t}$\;
}
\Return $X_t$
\caption{FAST\large SLAM}
\end{algorithm}

\section{Obrázky}

Do našich článků můžeme samozřejmě vkládat obrázky. Pokud je obrázkem fotografie,
můžeme klidně použít bitmapový soubor. Pokud by to ale mělo být nějaké schéma nebo
něco podobného, je dobrým zvykem takovýto obrázek vytvořit vektorově. 

\begin{figure}[h]
    \begin{center}
        \begin{tabular}{l}
            \includegraphics[scale=0.45]{pictures/etiopan.eps}
            \reflectbox{\includegraphics[scale=0.45]{pictures/etiopan.eps}}
        \end{tabular}
    \end{center}
\caption{Malý Etiopánek a jeho bratříček}
\label{obr:1}
\end{figure}

Rozdíl mezi vektorovým \dots

\begin{figure}[h]
    \centering
    \includegraphics[scale=0.5]{pictures/oniisan.eps}
    \caption{Vektorový obrázek}
    \label{obr:2}
\end{figure}

\noindent
\dots a bitmapovým obrázkem

\begin{figure}[h]
    \centering
    \includegraphics[scale=0.75]{pictures/oniisan2.eps}
    \caption{Bitmapový obrázek}
    \label{obr:3}
\end{figure}

\noindent
se projeví například při zvětšení.

Odkazy (nejen ty) na obrázky \ref{obr:1}, \ref{obr:2} a \ref{obr:3}, na  
tabulky \ref{tab:1} a \ref{tab:2} a také na algoritmus \ref{algorithm:1} jsou udělány pomocí 
křížových odkazů. Pak je ovšem potřeba zdrojový soubor přeložit dvakrát.

Vektorové obrázky lze vytvořit i přímo v LATEXu, například pomocí prostředí 
\verb!picture!.

\newpage
\begin{landscape}
\begin{figure}
    \setlength{\unitlength}{1cm}
    \thicklines
    \begin{picture}(23,15)
        \put(0,0){\framebox(25,15){}}
        %frame of house
        \put(3,2){\line(1,0){17}}
        \put(3,2){\line(0,1){8}}
        \put(20,2){\line(0,1){8}}
        \put(2,10){\line(1,0){19}}
        %roof
        \put(2,10){\line(3,2){7.5}}
        \put(21,10){\line(-3,2){7.5}}
        %windows
        \put(3,3){\line(1,0){4}}
        \put(5,3){\line(0,1){2}}
        \put(3,5){\line(1,0){4}}
        \put(7,3){\line(0,1){2}}
        \put(10,3){\line(1,0){2}}
        \put(10,3){\line(0,1){2}}
        \put(10,5){\line(1,0){2}}
        \put(12,5){\line(0,-1){2}}
        %door
        \put(16,2){\line(0,1){3}}
        \put(16,5){\line(1,0){1.5}}
        \put(17.5,5){\line(0,-1){3}}
        %right-top windows
        \put(15,6.5){\line(1,0){3}}
        \put(15,6.5){\line(0,1){2}}
        \put(15,8.5){\line(1,0){3}}
        \put(18,8.5){\line(0,-1){2}}
        %left-top window
        \put(7,6.5){\line(1,0){3}}
        \put(7,6.5){\line(0,1){2}}
        \put(7,8.5){\line(1,0){3}}
        \put(10,8.5){\line(0,-1){2}}
        %pavement
        \put(15.5,2){\line(1,-1){2}}
        \put(18,2){\line(1,-1){0.75}}
        \put(18.75,1.25){\line(1,0){2}}
        %garage
        \put(20,2){\line(1,-1){2}}
        \put(20,8){\line(1,-1){5}}
        \put(20,8){\line(1,0){5}}
        \put(21,1){\line(0,1){3}}
        \put(21,4){\line(1,-1){1}}
        \put(22,3){\line(0,-1){3}}
        %sun
        \put(22,13){\circle{1}}
        \put(22.5,13.5){\line(1,1){0.5}}
        \put(22.75,13){\line(1,0){0.70}}
        \put(22.5,12.5){\line(1,-1){0.5}}
        \put(22,12.25){\line(0,-1){0.60}}
        \put(21.5,12.5){\line(-1,-1){0.5}}
        \put(21.25,13){\line(-1,0){0.7}}
        \put(21.5,13.5){\line(-1,1){0.5}}
        \put(22,13,75){\line(0,1){0.6}}
        
    \end{picture}
    \caption{moderní rodinný dům}
    \label{obr:4}
\end{figure}

\end{landscape}


\end{document}
