\documentclass{article}
\usepackage[utf8]{inputenc}
\usepackage{hyperref}
\usepackage{url}
\DeclareUrlCommand\url{\def\UrlLeft{<}\def\UrlRight{>} \urlstyle{tt}}

\title{Obecná teorie relativity}
\author{Vojtěch Bůbela}
\date{April 2020}




\begin{document}

\maketitle

\section{Albert Einstein}
Obecnou teorii relativity Albert Eistein poprvé publikoval v roce 1915, 10 let po publikaci speciální teorie relativity, která, narozdíl od Obecné teorie relativity nezahrnovala gravitaci\cite{online_article1}. Publikací tohoto díla se Albert Einstein celosvětově proslavil. V roce 1999 ho časopis TIME prohlásil za osobu století a  označil ho jako ztělesnění inteligence nebo také jako bělovlasého staříka, jehož obličej znají snad všichni \cite{Times}. Albert Einstein není ovšem známý pouze za svoji teorii relativity, nechal si také totiž patentovat mnoho vynálezů, npaříklad ledničku a jiné.\cite{albert_patents} Albert Einstein se během svého života podílel na mnoha věcech a to i například na projektu Manhattan, jenž vedl k výrobě atomové bomby.\cite{manhattan}

\section{Obecná teorie relativity}


Světoznámá teorie, která také bývá také nazývána nejkrásnější teorii fyziky, byla dokázána během zatmění slunce v roce 1919.\cite{ccf} Obecná teorie relativity sdružuje speciální teorii relativity a Newtonův gravitační zákon.\cite{wikipedia} Do doby vydání teoreie relativity byl Isac Newton považován za boha kvůli jeho univerználnímu gravitačnímu zákonu,\cite{nadherna_teorie} avšak Albert Einstein ve své knize konstatuje, že Newtonoská gravitace platí pouze v Newtonovském systému.\cite{relativity} Einstein také Newtonovu teorii vylepšil, tím že ji zkombinoval s elektromagnetickými zákony.\cite{foto2} Mnoho lidí se může domnívat, že je to práve obecná teorie relativity, která Albertu Einsteinovi vyhrála Nobelovu cenu, není to však pravda, Nobelu cenu Albert Einstein obdržel za vysvětlení fotoelektrického efektu.\cite{foto}

\newpage
\bibliographystyle{czechiso}
\bibliography{proj4}
\end{document}
