\documentclass[11pt, twocolumn]{article}
\usepackage[czech]{babel}
\usepackage[utf8]{inputenc}
\usepackage{times}
\usepackage{amsthm}
\usepackage{amsmath}
\usepackage{amssymb}
\usepackage[IL2]{fontenc}
\usepackage{geometry}
\geometry{
a4paper,
total={18cm,25cm},
left=1.5cm,
top=2.5cm,
}

\theoremstyle{definition}
\newtheorem{defn}{\textbf{Definice}}

\newtheorem{lemm}{Věta}

\begin{document}
\begin{titlepage}
\begin{center}
\Huge
\textsc
{Fakulta informačních technologií\\ Vysoké učení technické v brně} \\[0.4em]
\vspace{\stretch{0.382}}
\LARGE
Typografie a~publikování -- 2. projekt\\[0.3em]
Sazba dokumentů a~matematických výrazů
\vspace{\stretch{0.618}}
\end{center}

\Large2020 \hfill Vojtěch Bůbela(xbubel08)
\end{titlepage}


\section*{Úvod} 

V~této úloze si vyzkoušíme sazbu titulní strany, matematických vzorců, prostředí a dalších textových struktur obvyklých pro technicky zaměřené texty (například rovnice (\ref{eq:2}) nebo Definice \ref{defn:2} na straně \pageref{defn:2}). Pro vytvoření těchto odkazů používáme příkazy \verb!\label!, \verb!\ref! a \verb!\pageref!.

Na titulní straně je využito sázení nadpisu podle optického středu s~využitím zlatého řezu. Tento postup byl probírán na přednášce. Dále je použito odřádkování se zadanou relativní velikostí 0.4em a 0.3em.

\section{Matematický text} 

Nejprve se podíváme na sázení matematických symbolů a výrazů v~plynulém textu včetně sazby definic a vět s~využitím balíku \verb!amsthm!. Rovněž použijeme poznámku podčarou s~použitím příkazu \verb!\footnote!. Někdy je vhodnépoužít konstrukci \verb!${}$! nebo \verb!\mbox{}! která říká, že (matematický) text nemá být zalomen. V~následující de-finici je nastavena mezera mezi jednotlivými položkami\verb!\item! na 0.05em.

\begin{defn} \label{defn:1}
Turingův stroj \textsl{(TS) je definovám jako šestice tvaru M = (${Q, \Sigma, \Gamma, \delta, q_0, q_F}$), kde:}
\begin{itemize}
\setlength\itemsep{0.05em}

    \item \textsl{$Q$ je konečná množina} vnitřních (řídících) stavů,
    
    \item \textsl{$\Sigma$ je konečná množina symbolů nazývaná} vstupní abeceda, ${\Delta \notin \Sigma}$
    
    \item \textsl{$\Gamma$ je konečná množina symbolů, $\Sigma \subset \Gamma, \Delta \in \Gamma,$ nazývaná} pásková abeceda,
    
    \item $\delta$ : \textsl{($Q \backslash \{q_F\}) \times \Gamma \rightarrow Q\times(\Gamma \cup \{L,R\}), kde L, R \notin \Gamma,$ je parciální} přechodová funkce, a
    
    \item ${q_0 \in Q}$ \textsl{je} počáteční stav \textsl{a ${q_f \in Q}$} je koncový stav.
    
\end{itemize}
\end{defn}

Symbol $\Delta$ značí tzv. \textsl{blank} (prázdný symbol), který se vyskytuje na místech pásky, která nebyla ještě použita.
	
\textsl{Konfigurace pásky} se skládá z~nekonečného řetězce, který reprezentuje obsah pásky a pozice hlavy na tomto řetězci. Jedná se o~prvek množiny ${\{\gamma\Delta^\omega | \gamma \in \Gamma^*\} \times \mathbb{N}}$\footnote{Pro libovolnou abecedu $\Sigma$ je $\Sigma^\omega$ množina všech \textsl{nekonečných} řetězců nad $\Sigma$, tj. nekonečných posloupností symbolů ze $\Sigma$.}. \textsl{Konfigurační pásky} obvykle zapisujeme jako $\Delta xyz\underline{z}x\Delta ...$ (podtržení značí pozici hlavy). \textsl{Konfigurace stroje} je pak dána stavem řízení a konfigurační pásky. Formálně se jedná o~prvek množiny ${Q \times \{\gamma\Delta^\omega | \gamma \in \Gamma^* \} \times \mathbb{N}}$.

\subsection{Podsekce obsahující větu a odkaz}

\begin{defn} \label{defn:2}
Řetězec $\omega$ nad abecedou $\Sigma$ je přijat TS \textsl{M jestliže M při aktivaci z~počáteční konfigurace pásky $\underline{\Delta} \omega \Delta ...$~a~počátečního stavu $q_0$ zastaví přechodem do koncového stavu $q_f$, tj ($q_0,\Delta \omega \Delta^\omega, 0$) pro nějaké $\underset{M}{\overset{*}{\vdash}}(q_f, \gamma, n)$ pro nějaké $\gamma \in \Gamma^*$ a $n \in \mathbb{N} $.}

\textsl{Množinu ${L(M) = \{ \omega | \omega }$ je přijat TS M \}${\subseteq \Sigma^*}$ nazýváme} jazyk přijímaný TS M.
\end{defn}

Nyní si vyzkoušíme sazbu vět a důkazů opět s~použitím balíku \verb!amsthm!.

\begin{lemm} \label{lemm}
\textsl{Třída jazyků, které jsou přijímány TS, odpovídá }rekurzivně vyčíslitelným jazykům.
\end{lemm}

\begin{proof}
[Důkaz.] V~důkaze vyjdeme z~Definice 1 a 2.
\end{proof}

\section{Rovnice}

Složitější matematické formulace sázíme mimo plynulý text. Lze umístit několik výrazů na jeden řádek, ale pak je třeba tyto vhodně oddělit, například příkazem \verb!\quad!.

\begin{equation*}
    \sqrt[i]{x_i^3} \quad
    \mbox{ kde } x_i \mbox{ je }i \mbox{-te sude cislo } \quad
    y_i^{2-y_i} \neq y_i^{y_i^{y_i}}
\end{equation*}

V~rovnici (\ref{eq:1}) jsou vužity tři typy závorek s~různou explicitně definovanou velikostí.

\begin{align}
    x\quad&=\quad\Bigg\{\bigg([a + b] * c\bigg)^d \oplus 1\Bigg\} \label{eq:1} \\
    y\quad&=\quad\lim_{x\to\infty} \frac{\text{sin}^2 x + \text{cos}^2 x}{\frac{1}{\text{log}_{10} x}}
    \label{eq:2}
\end{align}



V~této větě vidíme, jak vypadá implicitní vysázení limity $\lim_{x\to\infty} f(n)$ v~normálním odstavci textu. Podobně je to i s~dalšími symboly jako $\sum_{i=1}^n 2^i$ či $\bigcap_{A \in \mathcal{B}}A$. V~případě vzorců $\lim\limits_{n \to \infty} f(n)$ a $\sum\limits_{i=1}^n 2^i$ jsme si vynutili méně úspornou sazbu příkazem \verb!\limits!.
	
\section{Matice}

Pro sázení matic se velmi často používá prostředí \verb!array! a závorky (\verb!\left!, \verb!\right!).

\[ \left(
\begin{array}{ccc}
a + b& \widehat{\xi + \omega}& \hat{\pi} \\
\vec{\mathbf{a}}& \overset{\longleftrightarrow}{AC}& \beta
\end{array}{}
\right) = 1 \Longleftrightarrow \mathbb{Q} = \mathcal{R}\]

Prostředí \verb!array! lze úspěšně využít i jinde.

\begin{equation*}
   \binom{n}{k} = \Bigg\{
\begin{array}{c l}
    0  &\text{pro } k~< 0 \text{ nebo } k~> n \\
    \frac{n!}{k!(n-k)!} &\text{pro } 0 \leq  k~\leq n.
\end{array}
\end{equation*}
\end{document}
